
\begin{center}
  \textbf{Похибку у роботі обчислюємо наступним чином:}
\end{center}
\begin{enumerate}
  \item Обчислюємо стандартну похибку за формулою $S_x = \sqrt{\frac{\sum\limits_{i=1}^n{x_i - x}^2}{n(n-1)}}$
  \item Обчислюємо випадкову похибку $\Delta x_{вип} = t(\alpha , n) S_x$,$t(\alpha, n )$ - коефіцієнт Стьюдента.
  \item $\Delta x_{instr} = 0,05\%$
  \item {$\Delta x = \sqrt{(\Delta x_{вип})^2 + (\Delta x_{instr})^2}$}
  \item $F_{рез} = F_{ser} \pm \Delta F$
\end{enumerate}
\subsection{Отримані результати у роботі}
Концентрація солі невідомого розчину $\% = 3,2 \pm 0,1$ \\
Питома рефракція для суміші розчинів $R = 0,000273 \pm 0,00002$\\
\section{Висновок}
\setlength{\parindent}{4em}
\qquad ми провели низку експериментів, у яких визначили коефіціент заломлення для розчинів з різною концентрацією солей, а також питому рефракцію для суміші розічинів. Отримані нами експериментальні результати є досить близькими до теоретичних, що мое свідчити про правильність виконання експериментів. Відповідні отримані результати наведені згори.\\
Відносна похибка:\\ $\xi_1 =\frac{0,1}{3,2} = 3\%$ \\
$\xi_2 =\frac{0,00002}{0,000273} = 7,3\%$
