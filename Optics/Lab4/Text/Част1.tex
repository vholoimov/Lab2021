\section{Вступна частина}
\setlength{\parindent}{4em}
\indent \textbf{Мета роботи:} Перевірити закон обернених квадратів для точкового джерела світла, побудувати векторну
діаграму сили світла лампи розжарювання. \par
\textbf{Прилади:} Оптична лава, візуальний фотометр, секторний послаблювач, лампи розжарювання
\begin{center}
\textbf{\emph{Теоретичні відомості}}
\end{center}
\qquad Точковим вважається джерело світла, розмірами якого
можна знехтувати порівняно з відстанню між цим
джерелом і точкою спостереження. Точкове джерело в
математичному розумінні не має фізичного змісту,
оскільки, вважаючи, що воно випромінює кінцеву
кількість енергії, ми повинні припустити наявність
нескінченно великої густини енергії в самому джерелі.
Маючи скінченні розміри, реальне точкове джерело
світла в загальному випадку є \textbf{анізотропним}, тобто
випромінює неоднакову світлову енергію в різних
напрямках. \\
Для точкового джерела між силою світла джерела і освітленістю поверхні, що опромінюється
даним джерелом, справедливе співвідношення:
$$E = \frac{F}{S} = \frac{F cos(\alpha)}{S} = \frac{F cos(\alpha)}{\Omega r^2} = \frac{I}{r^2} cos(\alpha)$$
Де $r$ - відстань від джерела до поверхні $S$, $\alpha$ - кут між $r$ та нормаллю до поверхні $S$. Зазначене співвідношення має назву \textbf{закон обернених квадраті} $(E \approx \frac{1}{r^2})$. Закон обернених квадратів виконуватиметься тим точніше, чим менші розміри джерела порівняно з $r$.
\subsection{Перевірка закону обернених квадратів}
\qquad Розмістивши лампу розжарювання на відстані $r_1$ від фотометра, маємо:
$$E_1 = \frac{I_1}{r^{2}_1}cos(\alpha)$$
Після зменшення сили світла, маємо:
$$E_2 = \frac{I_2}{r^{2}_1} cos(\alpha)$$
Перемістимо лампу на відстань $r+2$, щоб виконувалась рівність $E_1$ = $E_2$. Маємо наступну рівність
$$\frac{I_2}{r^{2}_2} = \frac{I_1}{r^{2}_1}$$
