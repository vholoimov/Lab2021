\section{Висновки}
\qquad Ми дослідили явище утворення кілець Ньютона та встановили залежності радіусу кілець $r_m$ від довжини хвилі $\lambda$ та радіусу кривизни дзеркала $R$. Визначити радіус кривизни лінзи та довжину хвилі червоного світла вдалося з високою точністю, а виміряти довжину хвилі синього світла вдалося не так точно, вірогідно, експериментальні дані були неточні. Похибки виникають в процесі замірів, оскільки важко визначити де саме знаходиться максимум чи мінімум, а не просто якісь добре чи погано освітлені точки. В цілому, результати задовільні і співпадають з очікуваними.
