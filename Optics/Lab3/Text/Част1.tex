\section{Вступна частина}
\setlength{\parindent}{4em}
\indent \textbf{Мета роботи:} Ознайомлення з експериментальними методами дослідження дисперсії світла в речовині і,
зокрема, з методом заломлення в призмі. \par
\textbf{Прилади:} Гоніометр, скляна призма, плоскопаралельна скляна пластина, ртутна лампа, трансформатор,
настільна лампа
\begin{center}
\textbf{\emph{Теоретичні відомості}}
\end{center}
Дисперсія світла в деякій речовині – це залежність показника заломлення $n$ цієї речовини від
частоти $\nu$ (довжини хвилі $\lambda$) світла, або залежність фазової швидкості світлових хвиль у речовині
$V$ від їх частоти (довжини хвилі). Наслідком дисперсії світла є розкладання в спектр пучка білого
світла при його проходженні крізь призму.\\
Відносну дисперсію обраховують за формулою:
$$N = \frac{n_B + n_R}{n_Y + 1}$$
де $n_B$ показник заломлення для хвилі з довжиною $\lambda=486,1нм$ (синя лінія водню), $n_R$ показник
заломлення для хвилі з довжиною $\lambda=656,3нм$ (червона лінія водню, C), $n_Y$ – показник заломлення
для хвилі з довжиною $\lambda=589,3нм$ (середнє з двох довжин хвиль, які відповідають двом близьким
жовтим лініям натрію D). \\
У даній роботі визначення показників заломлення скла проводиться методом заломлення в призмі.
За цим методом із досліджуваної речовини (скла) виготовляють призму і спостерігають у ній
заломлення світла. Показник заломлення $n$ для хвилі з довжиною $\lambda$ визначається за формулою:
$$n = \frac{sin\frac{\phi + \delta}{2}}{sin\frac{\phi}{2}}$$ \\
Де $\phi$ – кут заломлення призми, а $\delta$ – кут найменшого відхилення променів даної довжини хвилі.
