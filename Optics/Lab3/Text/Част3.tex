\section{Висновок}
\qquadВ даній роботі було досліджено метод визначення показника заломлення ізотропної твердої
прозорої речовини за вимірюваним заломлюючому кута призми з даної речовини та по куту
найменшого відхилення параксіального променя, що пройшов через призму. Вимірювання
вказаних кутів проводиться за допомогою гоніометра ГС-5. Також було встановлено залежність
показника заломлення світла від довжини хвилі, що падає на межу розділу двох середовищ (скла
та повітря), тобто досліджено явище дисперсії світла. \\
Встановлене значення відносної дисперсії N = 1,257 \\
Згідно графіку № 3.1 показник заломлення має обернену залежність від довжини хвилі: найбільш
довгим хвилям (хвилі, що відповідають червоним лініям спектра) відповідають найменші значення
показника заломлення (1,6541-1,6550), найкоротшим хвилям (хвилям, що відповідають
фіолетовим лініям спектра) відповідають найбільші значення показника заломлення (1,6883-
1,6894).
