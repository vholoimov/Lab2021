\section{Вступна частина}
\setlength{\parindent}{4em}
\indent \textbf{Мета роботи:} ознайомитись з деякими методами визначення фокусних відстаней лінз, визначити фокусні відстані збірної та розсіючої лінз, положення головних площин, показнак заломлення матеріалу лінз. \par
\textbf{Прилади:} оптична лава, товста збиральна плоско-опукла лінза, тонка розсіюча лінза, освітлювалювач, в захисному кожусь якого вирізана стрілка, зорова труба.
\begin{center}
\textbf{\emph{Теоретичні відомості}}
\end{center}
\indent Лінзою в оптиці називається прозоре для світла тіло, обмежене двома поверхнями.
Пряма, що проходить через центри сферичних поверхонь лінзи, називається її
головною оптичною віссю. Відстань між вершинами поверхонь лінзи є її товщиною. Лінзи,
товщина яких досить мала в порівнянні з радіусами кривизни їх поверхонь, називаються тонкими
лінзами. Для товстих лінз ця умова не виконується.\par
Лінзу можна розглянути як систему двох заломлюючих поверхонь. Система, в якій
зберігається гомоцентричність пучків і зображення виявляється строго геометричне подібним
предмету, називається ідеальною оптичною системою. Як показує теорія, зображення предметів
за допомогою ідеальної оптичної системи може бути побудоване без докладного дослідження
ходу променів всередині систем. Для цього потрібно лише знати фокусну відстань і положення
головних площин.\par
Головними площинами ідеальної оптичної системи називаються спряжені площини,
лінійне збільшення для яких дорівнює $\beta = \pm 1$. У тонкій лінзі головні площини співпадають, і їх
перетин з оптичною віссю дає оптичний центр. Головні площини лінзи в залежності від форми
лінзи можуть знаходитись як всередині лінзи, так і зовні.\par
Промені, паралельні головній оптичній осі, заломлюючись у збірній лінзі,
перетинаються в точці, що лежить на оптичній осі і називається головним фокусом лінзи. Існує
передній головний фокус $F$ і задній головний фокус $F'$. Відстані від головних площин до
головних фокусів називаються фокусними відстанями. Згідно з правилом знаків для збірної лінзи $f<0,f'>0$, для розсіюючої -$f>0,''`<0$.Величина, обернена фокусній відстані лінзи,
називається оптичною силою лінзи. Одиницею виміру оптичної сили лінзи є діоптрія (дптр).
Оптичну силу в одну діоптрію має лінза, фокусна відстань якої дорівнює одному метру.\\
Оптична сила товстої лінзи може бути розрахована за формулою:
$$\Phi = \frac{1}{f`}=(n-1)(\frac{1}{R_1}-\frac{1}{R_2}) + \frac{d (n-1)^2}{n R_1 R_2}$$
де $f'$ - задня фокусна відстань лінзи, $R_1$ та $R_2$ - радіуси кривизни заломлюючих поверхонь, $n$ - показник заломлення матеріалу лінзи, $d$ - товщина лінзи. \\
Відстань від головних площин до спряжених точок $S$ $S'$ зв'язані формулою Гаусса:
$$\frac{1}{f'} = \frac{1}{S'} - \frac{1}{S}$$
\newpage
\begin{center}
  {\textbf{\emph{Визначення фокусної відстані тонкої збірної лінзи}}}
\end{center}
\indent При визначенні фокусної віддалі збірної лінзи найпростіше будо б застосувати формулу (2). Але Але оскільки положення оптичного центра лінзи не завжди легко визначити, значення $f'$ знаходять знаходять, користуючись методом переміщення лінзи (методом
Бесселя). \\
Якщо закріпити предмет (джерело світла) і екран на відстані $L(L>4f)$, то пересуваючи між ними лінзу, можна знайти два таких її положення, при яких на екрані
утворюється чітке зображення, в одному положенні збільшене,  а в другому – зменшене. При цьому положення лінзи будуть симетричні відносно точки A, що лежить на
середині відрізка між предметом і екраном.\\
Дійсно, позначивши відстань між двома положеннями лінзи через $l$ і скориставшись
рівнянням лінзи (2) з урахуванням правила знаків, запишемо для першого положення лінзи:
$$\frac{1}{f'}= \frac{1}{S_1 '} - \frac{1}{S_1} = \frac{1}{1 + S_2 '} - \frac{1}{-(L-l-S_2 ')} = \frac{L}{(l + S_2 ')(L-l-S_2 ')}$$
а для другого положення, відповідно:
$$\frac{1}{f'}=\frac{1}{S_2'}-\frac{1}{-(L-S_2')}=\frac{L}{S_2'(L-S_2')}$$
Прирівнюючи праві частини цих рівнянь, знайдемо відстань між другим положенням лінзи і
екраном:
$$S_2'=\frac{L-l}{2}$$
Відстань між першим положенням лінзи і предметом $S_1$ дорівнює:
$$S_1=-(L-l-S_2')=-\frac{L-l}{2}$$
Отже, перше положення лінзи знаходиться від предмета на такій же відстані, як друге
положення від зображення, а значить ці положення симетричні відносно точки $A$. Цього
висновку можна дійти також, скориставшись властивістю зворотності світлових променів.\\
Щоб одержати вираз для експериментального визначення фокусної відстані лінзи,
розглянемо одне з положень лінзи, наприклад, перше. Для нього значення спряжених відстаней:
$$S_1 = -\frac{L-l}{2} \quad \quad \quad S_1' = \frac{L+l}{2}$$ \\
Підставляючи ці величини у формулу (2), одержимо вираз для фокусної відстані
тонкої лінзи:
$$f' = \frac{SS'}{S-S'} = \frac{L^2 - l^2}{4L}$$ \\
Цей спосіб зручний тим, що експериментально вимірюються лише переміщення
тонкої лінзи l та відстань L між двома спряженими площинами, в яких розташовані предмет і
екран.
\newpage
\begin{center}
  {\textbf{\emph{Визначення фокусної відстані та положення головних площин товстої
збірної лінзи.}}}
\end{center}
Фокусну відстань товстої збірної лінзи визначають за способом Аббе (рис. 2). Нехай
предмет $y$ знаходиться на відстані $-X_1$
 від головного фокуса $F$ товстої збірної лінзи.
Зображення предмета має розмір $-y_1'$\\
Лінійне збільшення ${\beta}_1$ буде:
$${\beta}_1 = \frac{y_1'}{y}=-\frac{f}{X_1}$$
Якщо пересунути предмет $y$ в положення ,То лінійне збільшення буде:
$${\beta}_2 = \frac{y_2'}{y}=-\frac{f}{X_2}$$ \\
З цих формул можна отримати вираз для фокусної відстані:
$$f=-f'=\frac{X_2-X_2}{\frac{1}{{\beta}_1}-\frac{1}{{\beta}_2}}=\frac{\Delta y_1' y_2'}{y(y_2'-y_1')}$$
\begin{center}
  {\textbf{\emph{Визначення фокусної відстані тонкої розсіюючої лінзи.}}}
\end{center}
Визначення фокусної відстані розсіюючої лінзи ускладнюється тим, що зображення
дійсних предметів одержуються уявними і не можуть бути безпосередньо виміряні. Це
ускладнення можна усунути двома способами.\\
У першому способі додатково використовується збірна лінза. На початку досліду на
оптичній лаві розміщують лише одну збірну лінзу  і одержують на екрані дійсне
зображення предмета '
$A$ , яке служитиме уявним предметом для розсіюючої лінзи. По лінійці
оптичної лави відмічають його положення. Потім на шляху променів, що виходять із збірної
лінзи, розміщують досліджувану розсіюючу лінзу. Зображення предмета переміститься
тепер у більш віддалену точку
$A''$ . Відмічаючи по лінійці оптичної лави положення $A''$ і координату розсіюючої лінзи $C$ , визначають відстань $A'C$ та $A''C$ і за формулою обчислюють $f'$ розсіюючої лінзи.\\
У другому способі, крім збірної лінзи, використовується ще зорова труба. Якщо
зображення, що дається збірною лінзою, співпадає з переднім фокусом розсіюючої лінзи, то
після заломлення в ній промені вийдуть з лінзи паралельним пучком. Паралельність пучка можна
встановити за допомогою зорової труби, настроєної на нескінченність. Знаючи положення
розсіюючої лінзи, а також положення її головного фокуса, легко визначити фикусну відстань,
якщо лінза тонка. Якщо розсіююча лінза товста, даний метод дозволяє визначити лише
положення фокуса
