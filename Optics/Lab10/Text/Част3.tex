
\begin{center}
  \textbf{Похибку у роботі обчислюємо наступним чином:}
\end{center}
\begin{enumerate}
  \item Обчислюємо стандартну похибку за формулою $S_x = \sqrt{\frac{\sum\limits_{i=1}^n{x_i - x}^2}{n(n-1)}}$
  \item Обчислюємо випадкову похибку $\Delta x_{вип} = t(\alpha , n) S_x$,$t(\alpha, n )$ - коефіцієнт Стьюдента.
  \item $\Delta x_{instr} = 10''$
  \item {$\Delta x = \sqrt{(\Delta x_{вип})^2 + (\Delta x_{instr})^2}$}
  \item $F_{рез} = F_{ser} \pm \Delta F$
\end{enumerate}
\subsection{Отримані результати для фокусної відстані збиральної лінзи:}
\qquadДовжина хвилі $\lambda = 816 \pm 144$ нм \\
\qquadВідносна похибка для довжини хвилі: $\epsilon = \frac{144}{816} = 17\%$
\section{Висновок}
\setlength{\parindent}{4em}
\qquad З урахуванням похибки, отриманий нами результат потрапляє у теоретичний проміжок значень для довжини хвилі. У роботі ми дослідили інтерференційну картину для біпризми Френеля, еспериментально визначили заломлюючий кут призми та довжину хвилі. Відносна похибка для довжини хвилі, а також теоретичне відхилення для довжини смужок знаходяться у межах 20\%. Отримані відповідні значення:
\begin{enumerate}
  \item $\lambda = 816 \pm 144$ нм
  \item $\alpha = 0.002$
\end{enumerate}
