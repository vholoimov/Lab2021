\section{Вступна частина}
\setlength{\parindent}{4em}
\indent \textbf{Мета роботи:} засвоїти інтерференційний метод вимірювання довжини хвилі за допомогою
біпризми Френеля. \par
\textbf{Прилади:} оптична лава, джерело світла (ртутна лампа або газовий лазер), конденсор, щілинна діафрагма, світлофільтри, біпризма Френеля, окулярний мікрометр,
теодоліт.
\begin{center}
\textbf{\emph{Теоретичні відомості}}
\end{center}
\indent Як відомо, два незалежних джерела світла майже завжди некогерентні. Тому одержати інтерференцію від них практично неможливо. Проте можна здійснити багато схем дослідів, в
яких когерентні джерела створюються шляхом розділення світла від основного джерела на два пучки. Існує багато приладів, які дозволяють спостерігати інтерференцію хвиль у таких пучках. Була запропонована реалізація інтерференції за допомогою білінзи Бійє, дзеркал Ллойда,
біпризми Френеля.\\
 У роботі ми зупинимося на інтерференційній схемі з біпризмою Френеля. Біпризма являє собою дві призми з малими заломлюючими кутами (близько $30’$), накладеними одна на одну
 % Вставить картинку
 Пучок світла, який падає від щілини $S$, після заломлення в біпризмі розділіться на два пучки, що перекриваються. При цьому пучки поширюються так, начебто вони виходять з двох
різних зображень щілини $S_1$ та $S_2$ . Оскільки джерела $S_1$ та $S_2$ когерентні, то в просторі за біпризмою можна спостерігати інтерференційну картину, локалізовану в усій області перекривання пучків. На рис.1 промені $S_1C_1$ та $S_2C_2$ обмежують область, де має місце перекривання пучків, а тому й інтерференція. Дійсно, промінь $S_1C_1$ –граничній для променів, які проходять через верхню половину біпризми. Те ж саме стосується променя $S_2C_2$.
Інтерференційна картина має вигляд світлих та темних смуг. Знайдемо зв’язок між характеристиками біпризми, умовами досліду, та властивостями інтерференційної картини. Нехай
показник заломлення біпризми Френеля – $n$, заломлюючий кут - $\alpha$, відстань від джерела до біпризми – $l$, довжина хвилі - $\lambda$. Знайдемо число інтерференційних смуг $N$ та відстань між
темними або світлими смугами. У певному наближенні ( кут $\alpha$ малий ) можна вважати, що
джерело $S$ та його уявні зображення розташовані в одній площині.
 Промінь від джерела, що нормально падає на верхню грань біпризми, відхиляється під
кутом $\phi$ до нормалі к вихідної верхньої грані. Між кутами $\phi$ та $\alpha$ існує співвідношення:
$$nsin\alpha = sin\phi$$
З рис.2 можна знайти $\phi = \beta + \alpha$
Оскільки $a \approx 0$, маємо:
$$n\alpha = \beta + \alpha, \beta = \alpha(n-1)$$
Врешті, відстань між двома уявними джерелами:
$$d = 2h = 2l tg(\beta) = 2l \alpha(n-1)$$
%Risunok 2
Різниця ходу для деякої точки $M$:
$$\Delta = d\frac{x}{L+l} = \frac{2l(n-1)\alpha x}{L+l}$$
де $х$ – відстань від точки $М$ до точки $О$ – основи перпендикуляра, опущеного на екран із середини
відстані $d$ між джерелами $S_1$ та $S_2$.4 Для світлої інтерференційної смуги виконується умова $\Delta = k\lambda$ ,
де $k$ – порядок інтерференції. Відстань між двома сусідніми смугами (ширина смуги) дорівнює:
$$\Delta X = \frac{\lambda(L+l)}{2l(n-1)\alpha}$$
З рис.3 видно, що максимальна область перекривання пучків визначається відстанню між
точками М і М’:
$$MM' = 2X_{max} = \frac{Ld}{2l}$$
Число смуг, які можна спостерігати на екрані, буде таке:
$$N = \frac{MM'}{\Delta X} = \frac{Ld}{2l\Delta X} = \frac{4lL(n-1)^2 {\alpha}^2}{(l+L)\lambda}$$
\begin{center}
  {\textbf{\emph{Порядок виконання роботи}}}
\end{center}
\indent Для визначення $\Delta X$ та відстані між уявними джерелами $S_1$ та $S_2$ використовують окулярний мікрометр і теодоліт.\\
До початку виканання вимірювання необхідно правильно встановити всі прилади.\\
Для визначення довжини хвилі, яку пропускає світлофільтр, можна скористатися наступними формулами:
$$\Delta X = \frac{\lambda (L+l)}{d}$$
Звідси:
$$\lambda = \frac{d*\Delta X}{L+l}$$
Для визначення $\frac{d}{L+l}$ використовують теодоліт. З вимирів теодоліту:
$$\frac{d}{L+l = 2tg\frac{\phi}{2}} \approx \phi$$
У цьому разі формулу длоя довжини хвилі можна записати наступним так:
$$\lambda = \phi \Delta X$$
Для визначення кута $\alpha$ можна скористатися формулою:
$$\alpha = \frac{\lambda(L+l)}{2l(n-1)\Delta X}$$
$$N = \frac{L(L+l)\phi}{2l\Delta X}$$
\begin{center}
  {\textbf{\emph{Результати вимірювання}}}
\end{center}
