\section{Вступна частина}
\subsection{Об'єкт дослідження}
\setlength{\parindent}{4em}
\indent Біполярні та уніполярні транзистори, залежність від
часу їхня вольт-амперна характеристика.
\subsection{Мета}
\indent Дослідити вихідні характеристики транзисторів різних
типів.

\subsection{Методи досліджень}
\indent Одержання зображення ВАХ транзисторів на екрані двоканального
осцилографа, що працює в режимі характериографа. \\
Побудова ВАХ шляхом вимірювання певної кількості значень сили струму на
колекторі, що відповідають певним значенням напруги (для певної сили
струму бази або напруги) для біполярного транзистора та певної кількості
значень сили струму стоку, що відповідають певним значенням напруги (для
певних значень напруги між затвором і витоком) для польового транзистора,
подання результатів вимірів у вигляді графіків.

\newpage
\section{Теоретична частина}
\subsection{Термінологія}
\indent \textbf{Транзистор} — керований нелінійний елемент, на основі якого можна
створювати підсилювачі електричних сигналів. \\~\\
\qquad \textbf{Біполярний транзистор}  — це напівпровідниковий прилад з двома –
переходами, що взаємодіють між собою, та трьома виводами, підсилювальні
властивості якого зумовлені явищами інжекції (введення) та екстракції
(вилучення) неосновних носіїв заряду. \\~\\
\qquad \textbf{Вихідна вольт-амперна характеристика (ВАХ) біполярного
транзистора} а — це залежність сили струму колектора від напруги між
колектором та емітером при певному значенні струму бази (або
напруги між базою та емітером ) в схемі зі спільним емітером.\\~\\
\qquad \textbf{Польовий (уніполярний) транзистор} — це напівпровідниковий прилад,
підсилювальні властивості якого зумовлені струмом основних носіїв, що
течуть по провідному каналу, провідність якого керується зовнішнім
електричним полем.  \\~\\
\qquad \textbf{Польовий транзистор з керувальним електродом} — це польовий
транзистор, керування струмом основних носіїв у якому здійснюється за
допомогою –переходу, зміщеного у зворотному напрямі. \\~\\
\qquad \textbf{Вихідна вольт-амперна характеристика (ВАХ) польового транзистора} – це залежність сили струму стоку від напруги між стоком та витокомпри певному значенні напруги між затвором та витоком.\\~\\
\newpage
