\section{Вступна частина}
\subsection{Об'єкт дослідження}
\setlength{\parindent}{4em}
\indent Підсилювальні каскади
\subsection{Мета}
\indent Виміряти коефіцієнти передачі за напругою
підсилювальних каскадів різних типів для гармонічних і імпульсних
вхідних сигналів, а також зсуви фаз між вихідними і вхідними сигналами.


\subsection{Методи досліджень}
\indent Метод співставлення: одночасне спостереження
вхідного та вихідного сигналів на екрані двоканального осцилографа із
наступним вимірюванням і порівнянням їх параметрів.


\newpage
\section{Теоретична частина}
\subsection{Термінологія}
\indent \textbf{Біполярний транзистор } — це напівпровідниковий прилад з двома p-n–
переходами, що взаємодіють між собою, та трьома виводами, підсилювальні
властивості якого зумовлені явищами інжекції (введення) та екстракції
(вилучення) неосновних носіїв заряду.  \\~\\
\textbf{Польовий (уніполярний) транзистор} - це напівпровідниковий прилад,
підсилювальні властивості якого зумовлені струмом основних носіїв, що течуть
по провідному каналу, провідність якого керується зовнішнім електричним
полем.  \\~\\
\textbf{Польовий транзистор з керувальним електродом}  — е польовий транзистор,
керування струмом основних носіїв у якому здійснюється за допомогою p-n–
переходу, зміщеного у зворотному напрямі. . \\~\\
Підсилювач електричних сигналів – радіоелектронний пристрій, що перетворює
вхідний електричний сигнал, який являє собою залежність від часу напруги Uвх
(t) або струму Івх (t), у пропорційний йому вихідний сигнал Uвих (t) або Івих (t),
потужність якого перевищує потужність вхідного сигналу.\\
Підсилювальний каскад – підсилювач, який містить мінімальне число
підсилювальних елементів (1–2 транзистори) і може входити до складу
багатокаскадного підсилювача. \\
Коефіцієнт передачі за напругою Ku – відношення амплітуди вихідного напруги
підсилювача до амплітуди вхідної.
\newpage
