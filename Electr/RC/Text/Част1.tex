\section{Вступна частина}
\subsection{Об'єкт дослідження}
\setlength{\parindent}{4em}
\indent Пасивні лінійні чотириполюсники, перетворення сигналів при проходженні через
такі чотириполюсники.
\subsection{Мета}
\indent Дослідити зміну параметрів прямокутних імпульсів та гармонічних сигналів при
проходженні через пасивні лінійні чотириполюсники, опанувати методи вимірювання
амплітудно-частотних та фазо-частотних характеристик пасивних RC-фільтрів та їх перехідних характеристик

\subsection{Методи досліджень}
\indent Метод співставлення, тобто одночасного спостереження вхідного та вихідного сигналів на екрані двоканального осцилографа із наступним вимірюванням і порівнянням
їх параметрів;\\
Метод фігур Лісажу, який полягає у спостереженні на екрані двоканального
осцилографа замкнених кривих, які є результатом накладання двох коливань, що відбуваються у двох взаємно перпендикулярних напрямках (вхідний і вихідний сигнали
подаються на пластини горизонтального та вертикального відхилення осцилографа відповідно).

\newpage
\section{Теоретична частина}
\subsection{Термінологія}
\indent \textbf{Чотириполюсник} - це електричне коло (ділянка електричного кола) з чотирма
полюсами, зажимами, клемами або іншими засобами приєднання до нього інших електричних кіл чи ділянок електричних кіл. \\~\\
\textbf{Пасивний чотириполюсник} - це такий чотириполюсник, який не здатний збільшувати потужність вхідного сигналу за рахунок додавання енергії від якогось іншого джерела енергії (внутрішнього чи зовнішнього по відношенню до чотириполюсника). Потужність, що виділяється в елементі кола, підключеного до виходу такого чотириполюсника, менша за потужність, що споживається від джерела сигналу, підключеного до входу чотириполюсника.\\~\\
\textbf{Активний чотириполюсник} - дозволяє збільшувати потужність вихідного сигналу порівняно з потужністю вхідного сигналу за рахунок внутрішніх або зовнішніх
джерел енергії. Має містити активний елемент.\\~\\
\textbf{Лінійний чотириполюсник} - це такий, для якого залежність між струмами, що течуть крізь нього, та напругами на його зажимах є лінійною. Такі чотириполюсники
складаються з лінійних елементів. \\~\\
\textbf{Лінійні елементи електричних кіл} - це такі елементи, параметри яких не залежать від величини струму, що протікає через них або від прикладеної до них напруги.
На виході лінійних чотириполюсників, на відміну від нелінійних, не можуть утворюватися гармоніки ( і т. д.) сигналу частоти , який подано на вхід.\\~\\
\textbf{Нелінійний чотириполюсник} - це такий, який містить нелінійні елементи.
Для нього згадані залежності між струмами та напругами при деяких їх величинах
перестають бути лінійними, а на виході можуть з’являтися гармоніки частот вхідних
сигналів\\~\\
\textbf{Пасивний фільтр} - це пасивний чотириполюсник, який містить реактивні
елементи (індуктивності, ємності), спад напруги на яких або струм через які залежить
від частоти, і завдяки цьому здатен перетворювати спектр сигналу, поданого на його вхід, шляхом послаблення певних спектральних складових вхідного сигналу. Решта
спектральних складових вхідного сигналу проходить через такий пасивний лінійний чотириполюсник, тобто він працює як фільтр для певних спектральних складових сигналу.
Фільтри, побудовані на конденсаторах і резисторах, називють RC-фільтрами.
\newpage
