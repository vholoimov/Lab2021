
\begin{center}
  \textbf{Контрольні запитання}
\end{center}
\qquad \textbf{1. Операційний підсилювач - } це
диференціальний підсилювач постійного струму, який в ідеалі має
нескінченний коефіцієнт підсилення за напругою і нульову вихідну напругу за
відсутності сигналу на вході, великий вхідний опір і малий вихідний, а також
необмежену смугу частот підсилюваних сигналів. Раніше такі високоякісні
підсилювачі використовувалися виключно в аналогових обчислювальних
пристроях для виконання математичних операцій, наприклад, складання та
інтегрування. Звідси і походить їх назва – операційні підсилювачі (ОП). \\
Робоча формула для операційного підсилювача:
$$V_{out} = (V_{+} - V_{-})\cdot G$$
\begin{itemize}
  \item $V_{out}$ - вихідна напруга.
  \item $V_{+}$ - напруга на неінвертуючому вході
  \item $V_{-}$ -  напруга на інвертючому вході
  \item $G$ - коефіцієнт підсилення
\end{itemize}
\par
\textbf{2. Інвертуючий та неінвертуючий вхід} Два входу ОП - Инвертируючий і Неінвертуючий названі так за притаманними їм властивостями. Якщо подати сигнал на Инвертируючий вхід, то на виході ми отримаємо інвертований сигнал, тобто зрушений по фазі на 180 градусів - дзеркальний. Якщо ж подати сигнал на Неінвертуючий вхід, то на виході ми отримаємо фазово незмінений сигнал.\\
Негативний зворотний зв'язок (НЗЗ) — тип зворотного зв'язку, при якому вихідний сигнал передається назад на вхід для погашення частини вхідного сигналу.\\
Позитивний зворотний зв'язок (ПЗЗ) — тип зворотного зв'язку, при якому вихідний сигнал передається назад на вхід для збільшення вхідного сигналу.\\
Нехай вхідний сигнал $u$ та вихідний сигнал $U$ підсилювача зв'язано лінійним співвідношенням:
$$U = ku$$
Якщо на вхід системи подати крім сигналу u ще й частково сигнал з виходу, так що загальний вхідний сигнал стане $$U = k(u + \alpha U)$$  — $\alpha$ певний коефіцієнт зворотного зв'язку. \\
\textbf{3. Зворотний зв’язок за напругою і за струмом}
Якщо коло зворотного зв'язку вмикається до виходу підсилювача паралельно його навантаженню $R_{н}$,то напруга зворотного зв'язку буде прямо пропорційною напрузі на виході. Такий зв'язок називають зворотним зв'язком за напругою.\\
Якщо коло ззворотного зв'язку увімкнено до входу підсилювача послідовно з його навантаженням, то напруга зворотнього зв'язку буде пропорційною струму в навантаженні $R_{н}$. Такий зворотній зв'язок називають зворотнім зв'язок за струмом.\\
Якщо коло зворотного зв'язку вмикається до входу підсилювача послідовно з джерелом вхідного сигналу, то зворотний зв'язок називають послідовним.\\
Якщо коло зворотного зв'язку вмикається до входу паралельно джерела сигналу, то зворотній зв'язок називають паралельним.

\section{Використані джерела}

\qquad Методичні вказівки до практикуму «Основи радіоелектроніки»
для студентів фізичного факультету / Упоряд. О.В.Слободянюк,
Ю.О.Мягченко, В.М.Кравченко.- К.: Поліграфічний центр «Принт
лайн», 2007.- 120 с.

\qquad Ю.О. Мягченко , Ю. М . Дулич , А.В.Хачатрян “Вивчення
радіоелектронних схем методом комп’ютерного моделювання” :
Методичне видання. – К.: 2006.- с.
