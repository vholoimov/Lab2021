\section{Вступна частина}
\setlength{\parindent}{4em}
\indent \textbf{Мета роботи:} ознайомитися з властивостями операційних
підсилювачів, опанувати способи підсилення електричних сигналів схемами з
ОП, охопленим негативним зворотним зв`язком та способи виконання
математичних операцій за допомогою схем з ОП. \par

\begin{center}
\textbf{\emph{Теоретичні відомості}}
\end{center}
\qquad \textbf{Операційний підсилювач - } це
диференціальний підсилювач постійного струму, який в ідеалі має
нескінченний коефіцієнт підсилення за напругою і нульову вихідну напругу за
відсутності сигналу на вході, великий вхідний опір і малий вихідний, а також
необмежену смугу частот підсилюваних сигналів. Раніше такі високоякісні
підсилювачі використовувалися виключно в аналогових обчислювальних
пристроях для виконання математичних операцій, наприклад, складання та
інтегрування. Звідси і походить їх назва – операційні підсилювачі (ОП).\par
\textbf{Створення зворотного зв`язку} полягає в тому, що частина вихідного
сигналу підсилювача повертається через ланку зворотного зв`язку (ЗЗ) на його
вхід. Якщо сигнал зворотного зв`язку подається на вхід у протифазі до вхідного
сигналу (різниця фаз $\Phi = 180^o$), то зворотний зв`язок називають негативним
(НЗЗ). Якщо ж він подається на вхід у фазі до вхідного сигналу
($\Phi = 0^o$), то такий зворотний зв`язок називають позитивним (ПЗЗ) \\
Основною інтегральною мікросхемою для створення аналогових
електронних пристроїв є операційний підсилювач (ОП). ОП являє собою мікросхему, що за своїми розмірами і ціною практично не відрізняється від
окремого транзистора, хоча вона й містить кілька десятків транзисторів, діодів і
резисторів.\\
Завдяки практично ідеальним характеристикам ОП реалізація на їх основі
різних схем виявляєьться значно простішою і дешевшою, ніж на окремих
транзисторах і резисторах.
Операційним підсилювачем називають багатокаскадний диференціальний
підсилювач постійного струму, який має в діапазоні частот до кількох десятків
кілогерц коефіцієнт підсилення більший за $10^4$
і за своїми властивостями
наближається до уявного «ідеального» підсилювача. Під «ідеальним»
розуміють такий підсилювач, який має:
\begin{enumerate}
  \item нескінченний коефіцієнт підсилення за напругою диференціального
вхідного сигналу ($K \to \infty$);
  \item нескінченний вхідний імпеданс ($Z_{вх} \to \infty$);
  \item  нульовий вихідний імпеданс ($Z_{вих} = 0$);
  \item  рівну нулеві напругу на виході ($_U{вих} = 0$) при рівності напруг на вході
($U_{вх1} = U_{вх2}$);
  \item нескінченний діапазон робочих частот.
\end{enumerate}

Характеристики реального ОП не такі ідеальні, як хотілося б. Однак, для
практичних цілей ці характеристики близькі до ідеальних: коефіцієнт
підсилення для низьких частот (за постійним струмом) $К > 10^4$, вхідний опір $R_{вх} > 10^6 Ом$; вихідний опір $R_{вих} < 10^2$ Ом; коефіцієнт підсилення падає до 1 на частоті порядка $10^6$
Гц (1 МГц); напруга зміщення $U_{зм}$ (визначається як напруга,
яку потрібно подати на вхід ОП, щоб вихідна напруга стала рівною нулеві) для
більшості ОП не перевищує 10 мВ, а для прецизійних – 10 мкВ.
