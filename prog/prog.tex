\documentclass[a4paper,12pt]{article}

\usepackage{cmap}					% поиск в PDF
\usepackage[T2A]{fontenc}			% кодировка
\usepackage[utf8]{inputenc}			% кодировка исходного текста
\usepackage[english,russian]{babel}	% локализация и переносы
\usepackage[left=2cm,right=2cm,top=2cm,bottom=2cm,bindingoffset=0cm]{geometry}
\usepackage{graphicx}
\usepackage{float}%"Плавающие" картинки
\usepackage{wrapfig}%Обтекание фигур (таблиц, картинок и прочего)
\usepackage{mathtext}
\usepackage{mathtools}
\usepackage{indentfirst}
\setlength{\parskip}{0em}
\setlength{\parindent}{4em}


\author{Холоімов Валерій}
\title{1.1 Наш первый документ}
\date{\today}

\begin{document} % Конец преамбулы, начало текста.
\section{Про мене}
Привіт, мене звати Холоімов Валерій Вячеславович. Я кандидат на пост голови студентського парламенту фізичного факультету. \\
\hspace*{17mm}Останній рік я активно займаюсь волонтерством у студентській організації Erasmus Student Network Kyiv. Ми допомагаємо іноземним студентам адаптуватися в Україні, а також українським студентам, які планують поїхати за обміном.\\
\hspace*{17mm}В результаті цієї роботи я набув досвіду організації різноманітних івентів, розвитку Telegram та Instagram сторінок, знаходження та роботи з партнерами і роботи з іноземцями з усієї Європи.

\section{Загальні положення програми}
Найголовніша ціль, заради якої має працювати студентський парламент факультету - комфорт студентів. Гнучкий розклад, зручний простір для навчання, якісне проведення лекцій та семінарів, стипендіальне забезпечення, підтримка студентської науки, своєчасне інформування студентів про всі новини факультету.\\
\hspace*{17mm}Основною метою моєї програми є різнобічний розвиток студентів, а не лише в науковому напрямку, і покращення комунікації між студентським парламентом і студентами.\\ \hspace*{17mm}Івенти про те, як зробити власне \textbf{CV} кращим, як правильно користуватися \textbf{Linkedin}. Зустрічі з різноманітними людьми, яких будуть обирати студенти: психологи, сексологи, бізнесмени та волонтери. Саме це все та ще багато чого іншого сприятиме розвитку студентів у всіх напрямках.
\subsection{Підтримка Telegram бота}
Своєчасне обслуговування та  оновлення інформації у наявному боті. Створення feedback bota для надання швидкої та якісної допомоги студентам.
\subsection{Оцінювання якості освіти}
Опитування студентів щодо якості освіти, проведення семінарів та лекцій.\\ \textbf{Опитування кожні 2 місяця}, які даватимуть результати вже до кінця поточного семестру. Контроль за проведенням занять у дистанційному форматі.
\subsection{Стипендіальне забезпечення}
Контроль академічних рейтингів. Відслідковування та контроль нарахування додаткових балів. \textbf{Контроль} діяльності членів студентського парламенту факультета під час дистанційної форми навчання.
\subsection{Підтримка наукової діяльності}
Інформування студентів про можливості обміну між університетами. Створення окремого каналу для сповіщення про \textbf{можливості}. Підтримка студентів, що займаються науковою діяльністю.
\newpage
\subsection{Зближення студентського парламенту та студентів}
Для певної кількості студентів студентський парламент може здаватися чимось дуже далеким, що ніяк не пов'язано з їх студентським життям. Студентський парламент має частіше комунікувати з студентами, задля кращого розуміння їх бажань та потреб. Тому планується \textbf{більша кількість опитувань}, що будуть пов'язані не лише з якістю освіти.
\subsection{Культурно-мистеціка діяльність}
Проведення Дня Фізика, Дня Першокурсника, Містерія, проведення неформальних заходів для студентів. Проведення квартирників, стендапів, спортивних та  інтелектуальних ігор.
\subsection{Співпраця з СПФ/І, СПУ}
Кожен класний проєкт можна зробити ще кращим, якщо робити його не однією командою, а декількома. В мене вже є домовленості з кандидатами на пост голів СПФ, СПУ про майбутні спільні проєкти та івенти.
\subsection{Зручний простір}
Пошук спонсорів для створення коворкінгу на фізичному факультеті.
\subsection{Комфортна зона назовні факультету}
Створення більш комфортного місця для куріння біля факультету.







\end{document}
