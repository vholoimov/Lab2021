\documentclass[a4paper,12pt]{article}

\usepackage{cmap}					% поиск в PDF
\usepackage[T2A]{fontenc}			% кодировка
\usepackage[utf8]{inputenc}			% кодировка исходного текста
\usepackage[english,russian]{babel}	% локализация и переносы
\usepackage[left=2cm,right=2cm,top=2cm,bottom=2cm,bindingoffset=0cm]{geometry}
\usepackage{graphicx}
\usepackage{float}%"Плавающие" картинки
\usepackage{wrapfig}%Обтекание фигур (таблиц, картинок и прочего)
\usepackage{mathtext}
\usepackage{mathtools}
\usepackage{indentfirst}
\setlength{\parskip}{0em}
\setlength{\parindent}{4em}


\author{Холоімов Валерій}
\title{1.1 Наш первый документ}
\date{\today}

\begin{document} % Конец преамбулы, начало текста.
\section{Загальні положення програми}
\qquad Найголовніша ціль, заради якої має працювати студентський парламент факультету - комфорт студентів. Гнучкий розклад, зручний простір для навчання, якісне проведення лекцій та семінарів, стипендіальне забезпечення, підтримка студентської науки, своєчасне інформування студентів про всі новини факультету.
\subsection{Підтримка Telegram бота}
Своєчасне обслуговування та  оновлення інформації у боті.
\subsection{Оцінювання якості освіти}
Опитування студентів щодо якості освіти, проведення семінарів та лекцій. Контроль за проведенням занять у дистанційному форматі.
\subsection{Стипендіальне забезпечення}
Контроль академічних рейтингів. Відслідковування та контроль нарахування додаткових балів
\subsection{Підтримка науки}
Інформування студентів про можливості обміну між університетами. Створення окремого каналу для сповіщення про можливості. Підтримка студентів, що займаються науковою діяльністю.
\subsection{Зближення студентського парламенту та студентів}
Для певно кількості студентів студентський парламент може здаватися чимось дуже далеким, що ніяк не пов'язано з їх студентським життям. Студентський парламент має частіше комунікувати з студентами, задля кращого розуміння їх бажань.
\subsection{Культурно-мистеціка діяльність}
Проведення Дня Фізика, Дня Першокурсника, Містерія, проведення неформальних заходів для студентів.
\subsection{Зручний простір}
Пошук спонсорів для створення коворкінгу на фізичному факультеті.
\subsection{Комфортна зона назовні факультету}
Створення більш комфортного місця для куріння біля факультету.







\end{document}
